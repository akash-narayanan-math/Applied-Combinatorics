\documentclass[12pt]{article}

% Opening
\title{Applied Combinatorics Homework 5}
\author{Akash Narayanan}
\usepackage{amsmath, amsfonts, amssymb, amsthm, enumitem, tikz}
\usepackage{caption, subcaption, float}

% Problem counters
\newcounter{chapternumber}


% Problem environment
\theoremstyle{definition}
\newtheorem{problem-internal}{Problem}[chapternumber]
\newenvironment{problem}{
  \medskip
  \begin{problem-internal}
}{
\end{problem-internal}
}

% Solution environment
\newenvironment{solution}{
  \begin{proof}[Solution]
    \vspace{-8px}
    \setlength{\parskip}{4px}
    \setlength{\parindent}{0px}
}{
\end{proof}
}

\begin{document}

  \maketitle

  % Problem 7.11
  \setcounter{chapternumber}{7}
  \begin{problem}
    A school has 147 third graders. The third grade teachers have planned a special treat for the last day of school and brought ice cream for their students.
    There are three flavors: mint chip, chocolate, and strawberry.
    Suppose that 60 students like (at least) mint chip, 103 like chocolate, 50 like strawberry, 30 like mint chip and strawberry, 40 like mint chip and chocoloate, 25 like chocolate and strawberry, and 18 like all three flavors.
    How many students don't like any of the flavors available?
  \end{problem}

  \begin{solution}
    Using the Inclusion-Exclusion Principle, we can find how many students don't like any of the flavors as follows:
    \begin{align*}
      1000 &- 60 - 103 - 50 \\
           &+ 30 +  40 + 25 \\
           &- 18 \\
           &= 864
    \end{align*}
    That is, 864 students don't like any of the flavors.
  \end{solution}


  % Problem 7.5
  \setcounter{problem-internal}{4}
  \begin{problem}
    How many positive integers less than or equal to 1000 are divisible by none of 3, 8, and 25?
  \end{problem}

  \begin{solution}
    The number of positive integers less than or equal to \(n\) divisible by \(k\) is \(\lfloor{\frac{n}{k}}\rfloor\). In particular, this counts the number of multiples of \(k\) less than or equal to \(n\).

    Then we have
    \begin{displaymath}
      1000 - \lfloor{\frac{1000}{3}}\rfloor - \lfloor{\frac{1000}{8}}\rfloor - \lfloor{\frac{1000}{25}}\rfloor + \lfloor{\frac{1000}{3 \cdot 8}}\rfloor + \lfloor{\frac{1000}{3 \cdot 25}}\rfloor + \lfloor{\frac{1000}{8 \cdot 25}}\rfloor - \lfloor{\frac{1000}{3 \cdot 8 \cdot 25}}\rfloor = 560
    \end{displaymath}
    Thus, there are 560 positive integers less than or equal to 1000 that are divisible by none of 3, 8, and 25.
  \end{solution}


  % Problem 7.11
  \setcounter{problem-internal}{10}
  \begin{problem}
    Let \(X\) be the set of functions from \([n]\) to \([m]\) and let a function \(f \in X\) satisfy property \(P_{i}\) if there is no \(j\) such that \(f(j) = i\).
    \begin{enumerate}[label={\alph*.}]
      \item Let the function \(f:[8] \rightarrow [7]\) be defined by the table below.
      Does \(f\) satisfy property \(P_{2}\)? Why or why not?
      What about property \(P_{3}\)? List all the properties \(P_{i}\) with \(i \leq 7\) satisfied by \(f\).

      \begin{table}[H]
        \centering
        \begin{tabular}{c c c c c c c c c}
          \(i\) & 1 & 2 & 3 & 4 & 5 & 6 & 7 & 8 \\
          \hline
          \(f(i)\) & 4 & 2 & 6 & 1 & 6 & 2 & 4 & 2
        \end{tabular}
        \caption{A table defining a function}
      \end{table}

      \item Is it possible to define a function \(g:[8] \rightarrow [7]\) that satisfies no property \(P_{i}\) for \(i \leq 7\)?
      If so, give an example. If not, explain why not.

      \item Is it possible to define a function \(h:[8] \rightarrow [9]\) that satisfies no property \(P_{i}\) for \(i \leq 9\)? If so, give an example. If not, explain why.
    \end{enumerate}
  \end{problem}

  \begin{solution}
    \hfill
    \begin{enumerate}[label={\alph*.}]
      \item \(P_{2}\) states that there is no \(j\) such that \(f(j) = 2\). Looking at the table for \(f\), we can see that \(f(8) = 2\).
      Therefore, \(f\) does not satisfy \(P_{2}\).
      However, \(P_{3}\) is satisfied because there is no \(j\) such that \(f(j) = 3\). Furthermore, \(f\) satisfies \(P_{5}\) and \(P_{7}\).

      \item Yes, it is possible. Define \(g\) as follows:

      \begin{table}[H]
        \centering
        \begin{tabular}{c c c c c c c c c}
          \(i\) & 1 & 2 & 3 & 4 & 5 & 6 & 7 & 8 \\
          \hline
          \(f(i)\) & 1 & 2 & 3 & 4 & 5 & 6 & 7 & 7
        \end{tabular}
        % \caption{A table defining a function}
      \end{table}
      Then for all \(i \leq 7\), there exists a \(j\) such that \(g(j) = i\).

      \item No, it is not possible. Such a function \(h\) would have to be surjective. For a surjection to exist from \(A\) to \(B\), we need \(|A| \geq |B|\). However, \(|[8]| < |[9]|\) so such a function cannot exist. Namely, there is always at least one element of \([9]\) that does not get mapped to.
    \end{enumerate}
  \end{solution}


  % Problem 7.19
  \setcounter{problem-internal}{18}
  \begin{problem}
    How many derangements of a nine-element set are there?
  \end{problem}

  \begin{solution}
    By Inclusion-Exclusion, the number of derangements \(d_{n}\) on \(n\) elements is given by
    \begin{align*}
      d_{n} = \sum_{k = 0}^{n} (-1)^{k} {n \choose k} (n - k)!
    \end{align*}
    Letting \(n = 9\), we find that \(d_{n} = 133496\).
  \end{solution}


  % Problem 8.2 (a, c, d, f, i)
  \setcounter{chapternumber}{8}
  \setcounter{problem-internal}{1}
  \begin{problem}
    For each \textit{infinite} sequence suggested below, give its generating function in closed form, i.e. \textit{not} as an infinite sum.
    (Use the most obvious choice of form for the general term of each sequence.)
    \begin{enumerate}[label={\alph*.}]
      \item \(0, 1, 1, 1, 1, 1, \ldots\)
      \addtocounter{enumi}{1}
      \item \(1, 2, 4, 8, 16, 32, \ldots\)
      \item \(0, 0, 0, 0, 1, 1, 1, 1, 1, 1, 1, 1, 1, 1, 1, \ldots\)
      \addtocounter{enumi}{1}
      \item \(2^{8}, 2^{7} {8 \choose 1}, 2^{6} {8 \choose 2}, \ldots, {8 \choose 8}, 0, 0, \ldots\)
      \addtocounter{enumi}{2}
      \item \(3, 2, 4, 1, 1, 1, 1, 1, 1, \ldots\)
    \end{enumerate}
  \end{problem}

  \begin{solution}
    \hfill
    \begin{enumerate}[label={\alph*.}]
      \item The generating function \(f(x)\) takes the form
      \begin{align*}
        f(x) &= x + x^{2} + x^{3} + x^{4} \cdots \\
             &= x (1 + x + x^{2} + x^{3} \cdots)
      \end{align*}
      Recall that the generating function for the factor on the right is
      \begin{align*}
        1 + x + x^2 + x^3 + \cdots = \frac{1}{1 - x}
      \end{align*}
      Thus, we have
      \begin{align*}
        f(x) = \frac{x}{1 - x}
      \end{align*}

      \addtocounter{enumi}{1}

      \item The generating function \(f(x)\) takes the form
      \begin{align*}
        f(x) &= 1 + 2x + 4x^{2} + 8x^{3} + \cdots + (2x)^{n} + \cdots
      \end{align*}
      We can substitute this into the generating function referenced in part a and see that
      \begin{align*}
        f(x) = \frac{1}{1 - 2x}
      \end{align*}

      \item The generating function \(f(x)\) takes the form
      \begin{align*}
        f(x) &= x^{4} + x^{5} + x^{6} + x^{7} \cdots \\
             &= x^{4} (1 + x + x^{2} + x^{3} + \cdots)
      \end{align*}
      Once again, we know the generating function
      \begin{align*}
        1 + x + x^{2} + x^{3} + \cdots = \frac{1}{1 - x}
      \end{align*}
      Then we find the generating function for the sequence is
      \begin{align*}
        f(x) = \frac{x^{4}}{1 - x}
      \end{align*}

      \addtocounter{enumi}{1}

      \item The general term for each sequence is
      \begin{align*}
        2^{8 - n}{8 \choose n} = 2^{n} {8 \choose n}
      \end{align*}
      The generating function for \(8 \choose n\) is given by \((1 + x)^8\).
      This is because expanding the binomial means the coefficient of \(x^n\) involves choosing an \(x\) from the 8 factors \(n\) times.
      To include the \(2^n\), we simply replace the \(x\) with a \(2x\) in the generating function so that the coefficient of \(x^n\) has the \({8 \choose n}\) as well as a \(2^n\).
      Thus, the generating function for the sequence is
      \begin{align*}
        f(x) = (1 + 2x)^8
      \end{align*}

      \addtocounter{enumi}{2}

      \item For this sequence, we simply adjust the generating function for the sequence \(a_n = 1\). Indeed, we know
      \begin{align*}
        \frac{1}{1 - x} = 1 + x + x^2 + x^3 + \cdots
      \end{align*}
      Thus, we can add the necessary terms to adjust the coefficients of \(1\), \(x\), and \(x^2\).
      Then we have
      \begin{align*}
        f(x) = \frac{1}{1 - x} + 3x^2 + x + 2
      \end{align*}
    \end{enumerate}
  \end{solution}


  % Problem 8.7
  \setcounter{problem-internal}{6}
  \begin{problem}
    Consider the inequality
    \begin{equation*}
      x_{1} + x_{2} + x_{3} + x_{4} \leq n
    \end{equation*}
    where \(x_{1}, x_{2}, x_{3}, x_{4}, n \geq 0\) are all integers.
    Suppose also that \(x_{2} \geq 2\), \(x_{3}\) is a multiple of 4, and \(0 \leq x_{4} \leq 3\).
    Let \(c_{n}\) be the number of solutions of the inequality subject to these restrictions.
    Find the generating function for the sequence \(\{c_{n}: n \geq 0\}\) and use it to find a closed formula for \(c_{n}\).
  \end{problem}

  \begin{solution}
    The solution involves finding the generating function for each variable and multiplying them.
    We start by introducing a fifth variable \(x_5 \geq 0\) which will force the equality to hold.
    Each variable has a generating sequence depending on \(n\) and the constraints specified.
    \(x_1\) has no extra constraints so it generates the sequence
    \begin{align*}
      x_{1_{n}} = 1
    \end{align*}
    Thus, it is generated by the function
    \begin{align*}
      \sum_{n=0}^{\infty} x^n = \frac{1}{1 - x}
    \end{align*}
    \(x_2\) must be greater than or equal to 2, so the generating function is
    \begin{align*}
      \sum_{n=0}^{\infty} = \frac{x^2}{1 - x}
    \end{align*}
    which corresponds with all the powers of \(x\) greater than or equal to 2.
    \(x_3\) is a multiple of 4 so we consider the generating function
    \begin{align*}
      1 + x^4 + x^8 + x^{12} + \cdots = \frac{1}{1 - x^4}
    \end{align*}
    \(x_4\) is less than 4 so it is generated by
    \begin{align*}
      1 + x + x^2 + x^3 = \frac{1 - x^4}{1 - x}
    \end{align*}
    or the powers of \(x\) less than 4.
    Finally, \(x_5\) has the same constraints as \(x_1\) so it is generated by
    \begin{align*}
      \sum_{n = 0}^{\infty} x^n = \frac{1}{1 - x}
    \end{align*}
    We multiply these generating functions and find
    \begin{gather*}
      \frac{1}{1 - x} \cdot \frac{x^2}{1 - x} \cdot \frac{1}{1 - x^4} \cdot \frac{1 - x^4}{1 - x} \cdot \frac{1}{1 - x} \\
      = \frac{x^2}{(1 - x)^4}
    \end{gather*}
    We recall the generation function for the binomial coefficient \(C(n+r-1, r-1)\) for fixed \(r\) is given by
    \begin{align*}
      \frac{1}{(1 - x)^n}
    \end{align*}
    Then we have
    \begin{align*}
      \frac{x^2}{(1 - x)^4} &= \sum_{n = 0}^{\infty} {n + 3 \choose 3} x^{n+2} \\
      &= \sum_{n=2}^{\infty} {n + 1 \choose 3} x^n \\
      &= \sum_{n=0}^{\infty} {n + 1 \choose 3} x^n
    \end{align*}
    Thus, the generating function for \(c_n\) is given by
    \begin{align*}
      \frac{x^2}{(1 - x)^4}
    \end{align*}
    and the sequence is
    \begin{align*}
      c_n = {n + 1 \choose 3}
    \end{align*}
  \end{solution}


  % Problem 8.9
  \setcounter{problem-internal}{8}
  \begin{problem}
    What is the generating function for the number of ways to select a group of \(n\) students from a class of \(p\) students?
  \end{problem}

  \begin{solution}
    The generating function for \({p \choose n}\) is given by
    \begin{align*}
      f(x) = (1 + x)^p
    \end{align*}
    This is because the coefficient of \(x^n\) is the number of ways to select \(x\) \(n\) times from the \(p\) factors of the binomial expansion.
  \end{solution}


  % Problem 8.21 (a, b, c, d)
  \setcounter{problem-internal}{20}
  \begin{problem}
    For each exponential generating function below, give a formula in closed form for the sequence \(\{a_{n}: n \geq 0\) it represents.
    \begin{enumerate}[label={\alph*.}]
      \item \(e^{7x}\)
      \item \(x^{2} e^{3x}\)
      \item \(\frac{1}{1+x}\)
      \item \(e^{x^{4}}\)
    \end{enumerate}
  \end{problem}

  \begin{solution}
    Recall that the general form for an exponential generating function is
    \begin{align*}
      f(x) = \sum_{n = 0}^{\infty} a_{n} \frac{x^n}{n!}
    \end{align*}
    \begin{enumerate}[label={\alph*.}]
      \item The generating function \(e^x\) represents the sequence \(a_{n} = 1\). The sequence generated by \(e^{7x}\) is determined by its Taylor series which is shown below.
      \begin{align*}
        e^{7x} = \sum_{n=0}^{\infty} \frac{(7x)^{n}}{n!} = 1 + 7x + 7^2 \frac{x^2}{2!} + 7^3 \frac{x^3}{3!} + \cdots
      \end{align*}
      Thus, \(e^{7x}\) represents the sequence
      \begin{align*}
        a_{n} = 7^{n}
      \end{align*}

      \item The Taylor series for \(e^{3x}\) is
      \begin{align*}
        e^{3x} = \sum_{n = 0}^{\infty} \frac{(3x)^n}{n!}
      \end{align*}
      Multiplying by \(x^2\) yields
      \begin{align*}
        x^{2} e^{3x} &= \sum_{n = 0}^{\infty} \frac{3^n x^{n+2}}{n!} \\
        &= \sum_{n = 2}^{\infty} \frac{3^{n - 2} x^n}{(n - 2)!} \\
        &= \sum_{n = 2}^{\infty} \frac{n (n - 1) 3^{n-2} x^{n}}{n!} \\
        &= \sum_{n = 0}^{\infty} \frac{n (n - 1) 3^{n-2} x^{n}}{n!}
      \end{align*}
      Thus, the sequence generated for \(n \geq 0\) is
      \begin{align*}
        a_{n} = n (n - 1) 3^{n - 2}
      \end{align*}

      \item Consider the generating function
      \begin{align*}
        \frac{1}{1 - x} &= 1 + x + x^2 + x^3 + \cdots \\
        &= \sum_{n=0}^{\infty} x^n
      \end{align*}
      Substituting in \(-x\), we find the generating function
      \begin{align*}
        \frac{1}{1 + x} &= 1 + (-x) + (-x)^2 + (-x)^3 + \cdots \\
        &= \sum_{n=0}^{\infty} (-1)^n x^n \\
        &= \sum_{n=0}^{\infty} (-1)^n n! \frac{x^n}{n!}
      \end{align*}
      As an exponential generating function, this represents the sequence
      \begin{align*}
        a_n = (-1)^n n!
      \end{align*}

      \item Consider the Taylor series for \(e^x\) shown below.
      \begin{align*}
        e^{x} = \sum_{n = 0}^{\infty} \frac{x^n}{n!}
      \end{align*}
      Substituting in \(x^4\) for \(x\), we find the Taylor series becomes
      \begin{align*}
        e^{x^{4}} &= \sum_{n = 0}^{\infty} \frac{x^{4n}}{n!} \\
        &= \sum_{n = 0}^{\infty} \frac{(4n)!}{n!} \frac{x^{4n}}{(4n)!}
      \end{align*}
      We make this transformation so we can reindex the summation over only multiples of 4. That is,
      \begin{align*}
        e^{x^4} = \sum_{4 \mid n} \frac{n!}{(n / 4)!} \frac{x^n}{n!}
      \end{align*}
      Then the generated sequence is
      \begin{align*}
        a_n &= \frac{n!}{(n/4)!} \text{ when } 4 \mid n \\
        a_n &= 0 \text{ when } 4 \nmid n
      \end{align*}
    \end{enumerate}
  \end{solution}
\end{document}
